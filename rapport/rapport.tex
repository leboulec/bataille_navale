\documentclass[french]{article}
\usepackage[utf8]{inputenc}
\usepackage[T1]{fontenc}
\usepackage{babel}
\usepackage{hyperref}

\title{Mini-projet: Bataille Navale}
\author{Clément Roger, Clément Leboulenger}
\date{Janvier 2020}

\begin{document}
\pagenumbering{gobble}
\maketitle

\newpage
\tableofcontents

\newpage
\pagenumbering{arabic}

\section{Sujet}
Le sujet était présenté ainsi:
\begin{quotation}
	Réalisation d'un jeu de bataille navale pour un joueur. Dans ce projet, l'utilisateur joue contre la machine et doit torpiller sa flotte en un minimum de coup(\url{http://fr.wikipedia.org/wiki/Bataille_navale(jeu)}). Le programme placera aléatoirement sa flotte. On implémentera deux niveau de jeu: \emph{débutant} qui donnera une indication quand le missile tombe dans l'eau sur une case adjacente d'un navire ; \emph{expert} sans indication.	
\end{quotation}

\end{document}
